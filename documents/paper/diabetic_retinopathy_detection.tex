\documentclass{article}


\usepackage{arxiv}

\usepackage[utf8]{inputenc} % allow utf-8 input
\usepackage[T1]{fontenc}    % use 8-bit T1 fonts
\usepackage{hyperref}       % hyperlinks
\usepackage{url}            % simple URL typesetting
\usepackage{booktabs}       % professional-quality tables
\usepackage{amsfonts}       % blackboard math symbols
\usepackage{nicefrac}       % compact symbols for 1/2, etc.
\usepackage{microtype}      % microtypography
\usepackage{lipsum}

\title{A template for the \emph{arxiv} style}


\author{
  David S.~Hippocampus\thanks{Use footnote for providing further
    information about author (webpage, alternative
    address)---\emph{not} for acknowledging funding agencies.} \\
  Department of Computer Science\\
  Cranberry-Lemon University\\
  Pittsburgh, PA 15213 \\
  \texttt{hippo@cs.cranberry-lemon.edu} \\
  %% examples of more authors
   \And
 Elias D.~Striatum \\
  Department of Electrical Engineering\\
  Mount-Sheikh University\\
  Santa Narimana, Levand \\
  \texttt{stariate@ee.mount-sheikh.edu} \\
  %% \AND
  %% Coauthor \\
  %% Affiliation \\
  %% Address \\
  %% \texttt{email} \\
  %% \And
  %% Coauthor \\
  %% Affiliation \\
  %% Address \\
  %% \texttt{email} \\
  %% \And
  %% Coauthor \\
  %% Affiliation \\
  %% Address \\
  %% \texttt{email} \\
}

\begin{document}
\maketitle

\begin{abstract}
Diabetic retinopathy is an eye disease that can affect people suffering diabetes. It causes damage to the blood vessels of 
the eyes, deteriorates the eyesight and can lead in the worst case to blindness of the patient. It is important to 
detect the disease in an early stage to mitigate it as good as possible with an early treatment. Analyzing images of 
eyes and classify the severity of diabetic retinopathy is a challenging task that requires expert knowledge. To assist 
doctors and medical personnel, a classification model shall be trained to classify the severity automatically. 
\end{abstract}

% 1 %%%%%%%%%%%%%%%%%%%%%%%%%%%%%%%%%%%%%%%%%%%%%%%%%%%%%%%%%%%%%%%%%%%%%%%%%%%%%%%%%%%%%%%%%%%%%%%%%%%%%%%%%%%%%%%%%%%%%%
\section{Introduction}
Text about why diabetic retinopathy Detection

Dataset
% 2 %%%%%%%%%%%%%%%%%%%%%%%%%%%%%%%%%%%%%%%%%%%%%%%%%%%%%%%%%%%%%%%%%%%%%%%%%%%%%%%%%%%%%%%%%%%%%%%%%%%%%%%%%%%%%%%%%%%%%%
\section{Object Classification}
\subsection{Problem analysis}
To tackle the problem of diabetic retinopathy detection, several methods are possible. Because the dataset consists of 
ordinally scaled data of 5 classes, regression could be used to estimate the serverity of a case. In addition, a the problem 
can be handled as a classification problem after one-hot-encoding the labels. As a third option, one can define a threshold 
to define problematic diabetic retinopathy and non-problematic diabetic retinopathy and can handle the problem as a binary 
classification. Further, only binary and multiclass classification are anaylzed.

A binary classification has the advantage of higher accuracy, but lacks details, because the network only outputs 0 or 1
and no information about the exact serverity of the disease. Metrics are also easy to implement, because precision, 
recall and f1-score are standard implementations and nicely interpretable.

A multiclass classification has typically a lower accuracy, because the network needs to pick the right class among 
several classes. It provides the benefit or receiving richer information, i.e. the exact serverity of the disease.
Evaluating a multiclass classification problem becomes harder, because missclassifications can vary in their error.
Classifiying a class 1 as class 2 is for example less problematic than classifying class 1 as class 5.


\subsection{Architecture}
VGG, Resnet, Weight freeeze / unfreeze, GAP, Flatten, Dense Layers
\subsection{Weight initialization}
\subsection{Augmentation}
\subsection{Dataset Balancing}
\subsection{Training}
Adam, SGD, Momentum, Learing rate decay
\subsection{Metrics}
incl. QWC
% 3 %%%%%%%%%%%%%%%%%%%%%%%%%%%%%%%%%%%%%%%%%%%%%%%%%%%%%%%%%%%%%%%%%%%%%%%%%%%%%%%%%%%%%%%%%%%%%%%%%%%%%%%%%%%%%%%%%%%%%%
\section{Experiments}
\subsection{Procedure}
The established process for finding
WandB sweeps
\subsection{Hyperparameter selection}
\subsection{Grad cam}
% 4 %%%%%%%%%%%%%%%%%%%%%%%%%%%%%%%%%%%%%%%%%%%%%%%%%%%%%%%%%%%%%%%%%%%%%%%%%%%%%%%%%%%%%%%%%%%%%%%%%%%%%%%%%%%%%%%%%%%%%%
\section{Results}
best binary + multiclass performance;
color coded confusion matrix

%%%%%%%%%%%%%%%%%%%%%%%%%%%%%%%%%%%%%%%%%%%%%%%%%%%%%%%%%%%%%%%%%%%%%%%%%%%%%%%%%%%%%%%%%%%%%%%%%%%%%%%%
%%%%%%%%%%%%%%%%%%%%%%%%%%%%%%%%%%%%%%%%%%%%%%%%%%%%%%%%%%%%%%%%%%%%%%%%%%%%%%%%%%%%%%%%%%%%%%%%%%%%%%%%
%%%%%%%%%%%%%%%%%%%%%%%%%%%%%%%%%%%%%%%%%%%%%%%%%%%%%%%%%%%%%%%%%%%%%%%%%%%%%%%%%%%%%%%%%%%%%%%%%%%%%%%%
% below is just there for some help with the formatting
\section{Headings: first level}
\label{sec:headings}

\lipsum[4] See Section \ref{sec:headings}.

\subsection{Headings: second level}
\lipsum[5]
\begin{equation}
\xi _{ij}(t)=P(x_{t}=i,x_{t+1}=j|y,v,w;\theta)= {\frac {\alpha _{i}(t)a^{w_t}_{ij}\beta _{j}(t+1)b^{v_{t+1}}_{j}(y_{t+1})}{\sum _{i=1}^{N} \sum _{j=1}^{N} \alpha _{i}(t)a^{w_t}_{ij}\beta _{j}(t+1)b^{v_{t+1}}_{j}(y_{t+1})}}
\end{equation}

\subsubsection{Headings: third level}
\lipsum[6]

\paragraph{Paragraph}
\lipsum[7]

\section{Examples of citations, figures, tables, references}
\label{sec:others}
\lipsum[8] \cite{kour2014real,kour2014fast} and see \cite{hadash2018estimate}.

The documentation for \verb+natbib+ may be found at
\begin{center}
  \url{http://mirrors.ctan.org/macros/latex/contrib/natbib/natnotes.pdf}
\end{center}
Of note is the command \verb+\citet+, which produces citations
appropriate for use in inline text.  For example,
\begin{verbatim}
   \citet{hasselmo} investigated\dots
\end{verbatim}
produces
\begin{quote}
  Hasselmo, et al.\ (1995) investigated\dots
\end{quote}

\begin{center}
  \url{https://www.ctan.org/pkg/booktabs}
\end{center}


\subsection{Figures}
\lipsum[10] 
See Figure \ref{fig:fig1}. Here is how you add footnotes. \footnote{Sample of the first footnote.}
\lipsum[11] 

\begin{figure}
  \centering
  \fbox{\rule[-.5cm]{4cm}{4cm} \rule[-.5cm]{4cm}{0cm}}
  \caption{Sample figure caption.}
  \label{fig:fig1}
\end{figure}

\subsection{Tables}
\lipsum[12]
See awesome Table~\ref{tab:table}.

\begin{table}
 \caption{Sample table title}
  \centering
  \begin{tabular}{lll}
    \toprule
    \multicolumn{2}{c}{Part}                   \\
    \cmidrule(r){1-2}
    Name     & Description     & Size ($\mu$m) \\
    \midrule
    Dendrite & Input terminal  & $\sim$100     \\
    Axon     & Output terminal & $\sim$10      \\
    Soma     & Cell body       & up to $10^6$  \\
    \bottomrule
  \end{tabular}
  \label{tab:table}
\end{table}

\subsection{Lists}
\begin{itemize}
\item Lorem ipsum dolor sit amet
\item consectetur adipiscing elit. 
\item Aliquam dignissim blandit est, in dictum tortor gravida eget. In ac rutrum magna.
\end{itemize}


\bibliographystyle{unsrt}  
%\bibliography{references}  %%% Remove comment to use the external .bib file (using bibtex).
%%% and comment out the ``thebibliography'' section.


%%% Comment out this section when you \bibliography{references} is enabled.
\begin{thebibliography}{1}

\bibitem{kour2014real}
George Kour and Raid Saabne.
\newblock Real-time segmentation of on-line handwritten arabic script.
\newblock In {\em Frontiers in Handwriting Recognition (ICFHR), 2014 14th
  International Conference on}, pages 417--422. IEEE, 2014.

\bibitem{kour2014fast}
George Kour and Raid Saabne.
\newblock Fast classification of handwritten on-line arabic characters.
\newblock In {\em Soft Computing and Pattern Recognition (SoCPaR), 2014 6th
  International Conference of}, pages 312--318. IEEE, 2014.

\bibitem{hadash2018estimate}
Guy Hadash, Einat Kermany, Boaz Carmeli, Ofer Lavi, George Kour, and Alon
  Jacovi.
\newblock Estimate and replace: A novel approach to integrating deep neural
  networks with existing applications.
\newblock {\em arXiv preprint arXiv:1804.09028}, 2018.

\end{thebibliography}

\end{document}
